% !TEX encoding   = UTF8
% !TEX spellcheck = ru_RU

\documentclass[a4paper,11pt,landscape,notitlepage,oneside,openany,final]{memoir}

\input{../common/packages}

\usepackage{wasysym}

\input{../common/setup}
\input{../setup}

\renewcommand*\paperSubject{Успеваемость (\href{\latexurl}{\TeX}нический контроль)}

\input{../common/styles}
\input{../styles}



\begin{document}

\input{../common/title}


\newcommand{\tabheaderitem}[2]{\multicolumn{1}{#1}{\multirow{2}{*}{#2}}}

\newcommand*{\test}{\makebox[1.2cm]{тест}}
\newcommand*{\ctrl}[1]{\multicolumn{2}{c}{\makebox[1.2cm]{к./р. #1}}}
\newcommand*{\task}[1]{\makebox[0.8cm]{з. #1}}
\newcommand*{\quiz}[1]{\makebox[2.4em]{\small гл.\,#1}}
\newcommand*{\qsum}{\makebox{к./о.\(_\Sigma\)}}

\newcommand*{\groupmark}[1]{\textit{группа \textbf{#1}}}
\newcommand*{\groupsection}[1]{\AbstractSection{Группа №\,\textbf{#1}}}

\newcommand*{\pd}{\text{\(+\)\hspace{-3pt}.}}
\newcommand*{\md}{\text{\(-\)\hspace{-3pt}.}}

\newcommand*{\deadline}[1]{\textit{\bfseries\color{DarkRed!60}#1}}

\newcommand*{\grade}[1]{\textcolor{DarkRed!60}{(#1)}}


\clearpage
\begin{epigraphs}
    \large\itshape%
    \qitem{%
        Джамшида чашу я искал, не зная сна, \\
        Когда же мной земля была обойдена, \\
        От мужа мудрого узнал я, что напрасно \\
        Так далеко ходил, "--- в моей душе она.
    }{%
        Омар Хайям
    }
\end{epigraphs}

\begin{center}
    \begin{minipage}[b]{18em}
     	\large Омар Хайям был настоящим Учёным-энциклопедистом с~большой буквы. О~нём уважительно отзывались практически все его современники, называя его <<Учёнейшим мужем века>>, <<доказательством Истины>>, <<Имамом Хорасана>>, <<Царём философов Востока и Запада>>. Но самым главным его прозвищем, подчёркивающим его суть, было <<Мудрец, взрастивший в~сердце росток Любви Живой>>.

        \vspace{5ex}
    \end{minipage}\hspace{3em}
    \includegraphics[height=0.5\textheight]{images/omar_khayyam.png}
\end{center}


\clearpage
\renewcommand{\rightmark}{Задания}
Выполняя каждое задание, используйте систему контроля версий \git{}. Следите за~оформлением кода, выбирайте подходящие имена переменным и функциям. Тщательно готовьте тесты (тестовые сценарии игры, вычисления). Не~перекладывайте работу разработчика на~плечи проверяющего. Это поможет и вам самим убедиться в~работоспособности кода после внесённых изменений. Задавайте вопросы на~семинарах.

\textbf{NB!} Данные задания "--- это только повод для~беседы. В~процессе сдачи могут быть заданы дополнительные вопросы и подзадачи.



%%=====================
\subsection{Задание №1}
%%=====================
%%=====================
\subparagraph{Часть 1.}
%%=====================
\textit{Игра <<Ним>>}

\bigskip\noindent
\begin{minipage}[T]{0.58\columnwidth}\parindent=2.5em
    \emph{Ним} "--- одна из~самых старых и увлекательных математических игр. Для игры в~\emph{ним} необходим партнёр (в~\emph{ним} играют вдвоём), стол и набор фишек. В~качестве фишек обычно используются камешки или монетки. В~наиболее известном варианте \emph{нима} 12~фишек раскладываются в~три ряда так, как показано на~рисунке.

    Правила \emph{нима} просты. Игроки по~очереди забирают одну или несколько фишек из~любого ряда. Не~разрешается за~один ход брать фишки из~нескольких рядов. Выигрывает тот, кто возьмёт последнюю фишку (фишки).

    \smallskip

    Если вы сыграете несколько партий в~\emph{ним}, то скоро заметите, что существует некоторая оптимальная последовательность ходов, которая гарантирует победу, если только вы начинаете игру и первым ходом берёте две фишки из~первого ряда. Любой другой ход даст шанс вашему сопернику, который в~этом случае наверняка победит, если, в~свою очередь, воспользуется оптимальной стратегией.

    Полный анализ игры с~обобщением на~любое число рядов с~любым числом фишек в~каждом ряду впервые опубликовал в~1901~г. профессор математики из~Гарвардского университета Чарльз Л.\,Бутон (\textenglish{Charles L.\,Bouton}), который и назвал игру «ним» от~устаревшей формы английских глаголов «стянуть», «украсть».
\end{minipage}\hfill\begin{minipage}[T]{0.4\columnwidth}
    \includegraphics[width=\columnwidth]{images/nim_start_game.jpg}
\end{minipage}

\bigskip

\textbf{Разработайте программу}, которая будет выполнять роль партнёра в~игре, причём это будет весьма опасный противник, так как он будет «знать» оптимальную стратегию и умело ею пользоваться.

\medskip

\textbf{Срок сдачи}: не~позднее \deadline{30~сентября}.



%%=====================
\subparagraph{Часть 2.}
%%=====================
\textit{Калькулятор}

Доработайте окончательную версию калькулятора из~\textbookref{главы~7} учебника Бьярне Страуструпа, выполнив упражнения с~1-го по~9-е включительно.

Используйте <<тестовую оснастку>> для~автоматического тестирования программы. Для~этого подготовьте файл с~входными выражениями, как верными, так и ошибочными. Подавайте содержимое этого файла на~вход калькулятора, а~его вывод направьте в~другой файл. Проверьте выходные данные и запомните под~иным именем. При~повторном запуске программы после внесения изменений сравнивайте [программно] новый полученный файл с~проверенными ответами. Количество тестовых примеров для~финальной версии должно превышать~50.

Сгруппируйте исходный код логически, поместив каждую часть (функции, классы, их методы, константы) в~отдельный файл и связав её с~другими частями при~помощи заголовочного файла. Используйте идеи \textbookref{главы~8}.

\medskip

\textbf{Срок сдачи}: не~позднее \deadline{14~октября}.



%%==================================
\subsection{Контрольная работа №\,1}
%%==================================
%%==========================================
\subparagraph{Требования к оформлению кода:}
%%==========================================
\ldots



%%=======================
\subsection{Задание №\,2}
%%=======================
%%=====================
\subparagraph{Часть 1.}
%%=====================
\textit{Элементы графики "--- проект по~инженерному практикуму}

Разработайте версию небольшой полезной программы или простой логической игры на~\lang{C++} с~использованием элементов графики и графического пользовательского интерфейса на~основе библиотек \name{Graph\_lib} и \name{FLTK}.

Для~работы над~заданием объединитесь с~товарищами в~команду 3--4 человека. Утвердите выбор программы у~преподавателя. Распределите в~команде фронт работ.

Опишите кратко (тезисно) этапы создания программы (делайте это в~процессе работы над~заданием):

\begin{itemize}
    \item формулировка задачи;
    \item анализ задачи;
    \item идеи;
    \item проектирование;
    \item какие классы в~каком порядке создавались и редактировались;
    \item с~какими сложностями сталкивались в~процессе проектирования и реализации.
\end{itemize}

Оформите эти записи в~\href{https://ru.wikipedia.org/wiki/Markdown}{\lang{Markdown}} или в~простом текстовом файле. Можно использовать ключевые фрагменты кода.

\medskip

\emph{Ограничения}: программа должна быть без~анимации.

\medskip

\emph{Рекомендации}: приветствуется использование системы контроля версий. Для~удобства взаимодействия в~команде репозиторий следует разместить на~удалённом сервере, например, \name{GitHub}, \name{GitLab} и пр., или сервере ФАЛТа (за~доступом обращаться к~лектору).
Для~форматирования кода в~едином стиле, полезно настроить \code{Clang-format}.

\medskip

\emph{Требования}: задание считается выполненным, если
\begin{itemize}
    \item программа правильно выполняет задачу, для которой была сделана \grade{+8 баллов};
    \item программа протестирована исполнителем и минимум двумя товарищами из~двух других разных команд (предоставлены результаты тестирования) \grade{-1 балл};
    \item набор тестов достаточен \grade{-1 балл};
    \item исполнитель может объяснить любую часть исходного кода\footnote{Часть кода не~обязательно должна быть той, автором которой является исполнитель} \grade{-1 балл};
    \item исполнитель может исправить ошибку в~любой части исходного кода\footnotemark[\value{footnote}] \grade{-1 балл};
    \item в~исходном коде программы выдержан единый стиль (форматирования и именования) \grade{-1 балл};
    \item исполнитель выполнил дополнительные задания преподавателя \grade{+0..2 балла};
    \item исполнитель предоставил результаты тестирования программ двух других команд \grade{-1 балл};
    \item исполнитель предоставил свой вариант диаграммы классов и их интерфейсов для~программы другой команды \grade{-1 балл}.
\end{itemize}

\medskip

\textbf{Срок сдачи}: не~позднее \deadline{2~декабря}.


%%=====================
\subparagraph{Часть 2.}
%%=====================
\textit{Реализация вектора}

Запрограммируйте финальную версию вектора по~материалам \textbookref{главы~19} учебника Бьярне Страуструпа.

Напишите тестовый код, демонстрирующий работоспособность всего функционала. Используйте класс \code{Tracer} из~семинаров, если сочтёте необходимым.

\medskip

\textbf{Срок сдачи}: не~позднее \deadline{16~декабря}.



%%==================================
\subsection{Контрольная работа №\,2}
%%==================================
Основной нитью через всю контрольную проходит работа с~указателями и памятью непосредственно. Во~всех задачах будет включен контроль работы с~памятью, используя \code{valgrind}. То есть любые ошибки памяти, включая утечки, будут ошибками работы вашей программы.

%%==========================================
\subparagraph{Требования к оформлению кода:}
%%==========================================
такие же, как в~первой контрольной.



\newenvironment{results}[1][0pt]{
    \phantom{top of the page}
    \vspace{0pt plus 1fill}

    \noindent\hspace{#1}
    \begin{minipage}{\dimexpr\textwidth-#1\relax}
}{
    \end{minipage}

    \vspace{0pt plus 1fill}
    \phantom{bottom of the page}
}


\clearpage
\renewcommand{\rightmark}{Учебный план}
\begin{results}
    \input{data/curricula}
\end{results}


\clearpage
\renewcommand{\rightmark}{}
\begin{pycode}
import parse
import pathlib

import google_serve as gs


parse.set_rubicon(0.7)
datadir = pathlib.Path("data")
gc = gs.get_sheet(datadir/"token.json")
tabs = parse.read_json(datadir/"google_sheet.json")
for url in tabs["urls"]:
    title, table = parse.read_table(gc, url)
    print(rf"\groupsection{{{title}}}")
    print(rf"\renewcommand{{\rightmark}}{{\groupmark{{{title}}}}}")
    print("")
    print(r"\begin{results}")
    parse.print_main(table)
    print(r"\end{results}")
    print("")
    print(r"\clearpage")
    print(r"\begin{results}%[-0.5em]")
    parse.print_quiz(table)
    print(r"\end{results}")
    print("\n\n\n")
    print(r"\clearpage")
\end{pycode}

\end{document}
